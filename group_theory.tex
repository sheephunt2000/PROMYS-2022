\documentclass[letterpaper]{article}
\usepackage{fullpage}
\usepackage[top=2cm, bottom=4.5cm, left=2.5cm, right=2.5cm]{geometry}
\usepackage{amsmath,amsthm,amscd}
\usepackage{amsfonts}
\usepackage{amssymb}
\usepackage{lastpage}
\usepackage{enumerate}
\usepackage{fancyhdr}
\usepackage{xcolor}
\usepackage{graphicx}
\usepackage{listings}
\usepackage{hyperref}
\usepackage{enumitem}
\usepackage{XCharter}
\usepackage[T1]{fontenc}
\usepackage{textcomp}
\usepackage[euler-hat-accent]{eulervm}
\usepackage{eucal}
\usepackage{beton}
\usepackage{mathtools}
%\DeclareFontSeriesDefault[rm]{bf}{sbc}
\usepackage{xparse}
%\usepackage{comicsans}

\usepackage[nottoc,numbib]{tocbibind}

\usepackage{polynom}

\usepackage{faktor} % for quotient groups

\usepackage{bm}
\usepackage[b]{esvect}

\usepackage{csquotes}

\hypersetup{%
  colorlinks=true,
  linkcolor=blue,
  linkbordercolor={0 0 1}
}

\setlength{\parindent}{0.0in}
\setlength{\parskip}{0.05in}

% Edit these as appropriate
\newcommand\course{Math 452}
\newcommand\hwnumber{}                  % <-- homework numbern
\newcommand\studentname{Stephen Hu}           % <-- NetID of person #1
\newcommand\NetIDb{RUID: 192000354}           % <-- NetID of person #2 (Comment this line out for problem sets)

\pagestyle{fancyplain}
\headheight 30pt
\lhead{\studentname}
%\lhead{\studentname, Sangjun Ko}                 % <-- Comment this line out for problem sets (make sure you are person #1)
\chead{\textbf{\Large An Introduction to Group Theory}}
\rhead{PROMYS 2022}
\lfoot{}
\cfoot{}
\rfoot{\small\thepage}
\headsep 1.5em
\linespread{1.05} % line spacing
\parskip 6pt

\newtheorem{theorem}{Theorem}[section]
\newtheorem{lemma}[theorem]{Lemma}
\newtheorem{proposition}[theorem]{Proposition}
\newtheorem{corollary}[theorem]{Corollary}

\theoremstyle{definition}
\newtheorem{definition}[theorem]{Definition}
\newtheorem{example}[theorem]{Example}

\theoremstyle{remark}
\newtheorem{remark}[theorem]{Remark}

\newcommand\ZZ{\mathbb{Z}}
\newcommand\NN{\mathbb{N}}
\newcommand\RR{\mathbb{R}}
\newcommand\QQ{\mathbb{Q}}
\newcommand\CC{\mathbb{C}}
\newcommand\FF{\mathbb{F}}
\newcommand\VV{\mathbb{V}}
\newcommand\II{\mathbb{I}}
\renewcommand\AA{\mathbb{A}}
\renewcommand\aa{\alpha}
\newcommand\bb{\beta}
\newcommand\cc{\gamma}
\newcommand\card{\operatorname{card}}
\newcommand\rank{\operatorname{rank}}
\newcommand\im{\operatorname{im}}
\newcommand\sgn{\operatorname{sgn}}
\newcommand\gal{\operatorname{Gal}}
\newcommand{\defeq}{\coloneqq}
\newcommand{\SO}{\mathrm{SO}}
\newcommand{\GL}{\mathrm{GL}}
\newcommand{\SU}{\mathrm{SU}}
\newcommand{\SL}{\mathrm{SL}}
\DeclareMathOperator{\Mat}{Mat}
\newcommand{\ideal}{\vartriangleleft}
\newcommand{\parens}[1]{\left(#1\right)}
\newcommand{\brackets}[1]{\left[#1\right]}
\newcommand{\setbraces}[1]{\left\{#1\right\}}
\newcommand{\abrackets}[1]{\left\langle#1\right\rangle}
\DeclareMathOperator{\id}{id}
% \DeclareMathOperator{\sqrt}{rad}

\renewcommand{\vec}[1]{\vv{\boldsymbol{#1}}}



\begin{document}

\section{What's a group?}

At PROMYS, we have spent the bulk of our time investigating the ring \(\ZZ_m\)\footnote{This is known by most non-PROMYS mathematicians as \(\ZZ/m\ZZ{}\), notation which will make sense after this talk.}, with many interesting properties. But in some sense, rings are too complicated --- after all, they have two operations, addition and multiplication. If we look at just the addition properties, we can define a new mathematical structure which is important and interesting in its own right. 

\begin{definition}
	A \textbf{group} is a pair \((G,\ast)\), with \(\ast\) a binary operation on \(G\), such that the following properties hold: 
	\begin{itemize}
		\item \(G\) is \textbf{associative} under \(\ast\): for all \(a,b,c\in G\), \[(a\ast b)\ast c=a\ast(b\ast c).\]
		\item There exists an \textbf{identity element} \(e\in G\) such that \(e\ast a=a\ast e=a\) for all \(a\in G\). 
		\item For each element \(a\in G\), there exists an \textbf{inverse element} \(a^{-1}\in G\) such that \(a\ast a^{-1}=a^{-1}\ast a=e\). 
	\end{itemize}
\end{definition}

\begin{remark}
	Note we do not require \(\ast\) to be commutative (i.e. \(a\ast b=b\ast a\) for all \(a,b\in G\)). If it is, we say that \(G\) is \textbf{commutative} or \textbf{abelian}, after the Norwegian mathematician Niels Henrik Abel. 
\end{remark}

\begin{remark}
	This is a pedantic note, but some texts add an additional property that \(G\) be \textbf{closed} under \(\ast\). This is already implied from the definition of a binary operation, but for completeness' sake we mention it here. 
\end{remark}

\section{References}

The standard reference for anything abstract algebra is Dummit and Foote, but it's an admittedly huge, dense book which isn't the best for beginners. I would instead recommend Gallian, which has all around friendly exposition, as well as Artin, a very idiosyncratic book which I learned from and still use as a reference. It is probably a bit too wordy for PROMYS students, but for absolute beginners to proof-based math I would recommend Pinter. If you're feeling up to the task, go for Dummit and Foote. 

For a more unique approach, I would recommend Matt Macauley's notes here (\url{http://www.math.clemson.edu/~macaule/classes/s22_math4120/index.html}), which he is currently turning into a full-fledged textbook with projected completion by the end of 2022. His approach is far more visual, which he believes is the correct and more intuitive way to teach the subject, and I am excited to see his final product. This is definitely worth looking at if you are interested in another perspective than the one presented here and in other textbooks. 

I am also a fan of Evan Chen's Napkin: \url{https://web.evanchen.cc/napkin.html}. It is a more casual, succinct explanation of a good deal of modern mathematics, and I would recommend it to get a broad understanding and intuition for a topic, which can prove helpful to know for further research. As Evan Chen comes from an olympiad background, his style is biased towards that direction, but I still find the exposition quite helpful. 

% It turns out that the Nullstellensatz gives us a connection between algebraic varieties and quotient rings of the polynomial ring. This is where the name of the theorem comes from (in German, it means ``theorem of zeros.'').

% \begin{theorem}
% 	Let \(I\) be the ideal of \(\CC[x_1,\dots,x_n]\) generated by some polynomials \(f_1,\dots,f_r\), and let \(R\) be the quotient ring \(\CC[x_1,\dots,x_n]/I\). and let \(V\) be the variety given by these polynomials in \(\CC^n\). Then the maximal ideals of \(R\) are in bijective correspondence with the points of \(V\). 
% \end{theorem}

% \begin{proof}
% 	Note by the Correspondence Theorem we have that the maximal ideals of \(R\) are in one-to-one correspondence with the maximal ideals of \(\CC[x_1,\dots,x_n]\) that contain \(I\). Then a maximal ideal will contain \(I\) if and only if it contains the generators \(f_1,\dots,f_r\) of \(I\). But notice that every maximal ideal of \(\CC[x_1,\dots,x_n]\) is the kernel \(M_a\) of the substitution map \(x_i\mapsto a_i\) for some point \(a=(a_1,\dots,a_n)\). This means that the polynomials \(f_1,\dots,f_r\in M_a\) if and only if \(f_1(a)=\dots=f_n(a)=0\), i.e., if \(a\) is in the variety \(V\). 
% \end{proof}

% % there's a whole page at this point in Artin with a proof 
% % of a theorem involving AC and some corollaries of said theorem, 
% % but I'm not including it here because it doesn't
% % seem to have a connection to the rest of the section, and we
% % don't have a ton of room. we can put it back later if needed

% It's hard to visualize varieties in \(\CC^n\), so for the rest of this exposition, we will turn our focus to the special case of \(\CC^2\), and more specifically the polynomial ring \(\CC[t,x]\). 

% \begin{lemma}
% 	Let \(f(t,x)\) be a polynomial, and \(\alpha\in\CC{}\). Then the following are equivalent: 
% 	\begin{enumerate}[label=\emph{(\alph*)}]
% 		\item \(f(t,x)\) vanishes everywhere on the locus \(\{t=\alpha\}\) in \(\CC^2\). 
% 		\item \(f(\alpha,x)\) is the zero polynomial in \(x\). 
% 		\item \(t-\alpha\) divides \(f\) in \(\CC[t,x]\). 
% 	\end{enumerate}
% \end{lemma}
% \begin{proof}
% 	blah djshfjkhds
% \end{proof}

% Let \(\mathcal F\) be the field of rational functions \(\CC(t)\). Note that \(\CC[t,x]\) is a subring of \(\mathcal F[x]\), as we can write 
% \[f(t,x)=a_n(t)x^n+\cdots+a_1(t)x+a_0(t).\] Whenever we think about \(\CC[t,x]\), it is often helpful to investigate \(\mathcal F\) instead since it is a PID and we have access to the division algorithm. 

% \begin{proposition}
% 	Let \(h,f\) be non-zero elements of \(\CC[t,x]\) such that \(h\) is not divisible by any polynomial of the form \(t-\alpha\). Then \(h\mid f\) in \(\mathcal F[x]\) implies that \(h\mid f\) in \(\CC[t,x]\). 
% \end{proposition}
% \begin{proof}
% 	blahjwskdjskjdks
% \end{proof}

% \begin{theorem}
% 	Two non-zero polynomials \(f,g\in\CC[t,x]\) have finitely many zeros in \(\CC^2\), unless they have a common non-constant factor in \(\CC[t,x]\). 
% \end{theorem}
% \begin{proof}
% 	jldsfsjkfdsbksjfdb
% \end{proof}

% This theorem suggests that the varieties worth investigating in \(\CC^2\) are the locus of zeros of a single polynomial \(f(t,x)\). 

% % SANGJUN READ THIS 
% % I also doubt that the proofs above are particularly
% % relevant to the Riemann surface bs (the ones where)
% % I spammed in the proofs, i'm considering just
% % not including them
% % Read. -SK

% \begin{definition}
% 	The locus of zeros in \(\CC^2\) of a single polynomial \(f(t,x)\) is called the \textbf{Riemann surface} of \(f\). 
% \end{definition}

\end{document}