\documentclass[letterpaper]{article}
\usepackage{fullpage}
\usepackage[top=2cm, bottom=4.5cm, left=2.5cm, right=2.5cm]{geometry}
\usepackage{amsmath,amsthm,amscd}
\usepackage{amsfonts}
\usepackage{amssymb}
\usepackage{lastpage}
\usepackage{enumerate}
\usepackage{fancyhdr}
\usepackage{xcolor}
\usepackage{graphicx}
\usepackage{listings}
\usepackage{hyperref}
\usepackage{enumitem}
\usepackage{XCharter}
\usepackage[T1]{fontenc}
\usepackage{textcomp}
\usepackage[euler-hat-accent]{eulervm}
\usepackage{eucal}
\usepackage{beton}
\usepackage{mathtools}
%\DeclareFontSeriesDefault[rm]{bf}{sbc}
\usepackage{xparse}
%\usepackage{comicsans}

\usepackage[nottoc,numbib]{tocbibind}

\usepackage{polynom}

\usepackage{faktor} % for quotient groups

\usepackage{bm}
\usepackage[b]{esvect}

\usepackage{csquotes}

\hypersetup{%
  colorlinks=true,
  linkcolor=blue,
  linkbordercolor={0 0 1}
}

\setlength{\parindent}{0.0in}
\setlength{\parskip}{0.05in}

% Edit these as appropriate
\newcommand\course{Math 452}
\newcommand\hwnumber{}                  % <-- homework numbern
\newcommand\studentname{Stephen Hu}           % <-- NetID of person #1
\newcommand\NetIDb{RUID: 192000354}           % <-- NetID of person #2 (Comment this line out for problem sets)

\pagestyle{fancyplain}
\headheight 30pt
\lhead{\studentname}
\lhead{\studentname, Sangjun Ko}                 % <-- Comment this line out for problem sets (make sure you are person #1)
\chead{\textbf{\Large Final Project \hwnumber{}}}
\rhead{\course \\ \today}
\lfoot{}
\cfoot{}
\rfoot{\small\thepage}
\headsep 1.5em
\linespread{1.05} % line spacing

\newtheorem{theorem}{Theorem}[section]
\newtheorem{lemma}[theorem]{Lemma}
\newtheorem{proposition}[theorem]{Proposition}
\newtheorem{corollary}[theorem]{Corollary}

\theoremstyle{definition}
\newtheorem{definition}[theorem]{Definition}
\newtheorem{example}[theorem]{Example}

\theoremstyle{remark}
\newtheorem{remark}[theorem]{Remark}

\newcommand\ZZ{\mathbb{Z}}
\newcommand\NN{\mathbb{N}}
\newcommand\RR{\mathbb{R}}
\newcommand\QQ{\mathbb{Q}}
\newcommand\CC{\mathbb{C}}
\newcommand\FF{\mathbb{F}}
\newcommand\VV{\mathbb{V}}
\newcommand\II{\mathbb{I}}
\renewcommand\AA{\mathbb{A}}
\renewcommand\aa{\alpha}
\newcommand\bb{\beta}
\newcommand\cc{\gamma}
\newcommand\card{\operatorname{card}}
\newcommand\rank{\operatorname{rank}}
\newcommand\im{\operatorname{im}}
\newcommand\sgn{\operatorname{sgn}}
\newcommand\gal{\operatorname{Gal}}
\newcommand{\defeq}{\coloneqq}
\newcommand{\SO}{\mathrm{SO}}
\newcommand{\GL}{\mathrm{GL}}
\newcommand{\SU}{\mathrm{SU}}
\newcommand{\SL}{\mathrm{SL}}
\DeclareMathOperator{\Mat}{Mat}
\newcommand{\ideal}{\vartriangleleft}
\newcommand{\parens}[1]{\left(#1\right)}
\newcommand{\brackets}[1]{\left[#1\right]}
\newcommand{\setbraces}[1]{\left\{#1\right\}}
\newcommand{\abrackets}[1]{\left\langle#1\right\rangle}
\DeclareMathOperator{\id}{id}
\DeclareMathOperator{\rad}{rad}

\renewcommand{\vec}[1]{\vv{\boldsymbol{#1}}}
\renewcommand{\qedsymbol}{$\blacksquare$}



\begin{document}

\begin{displayquote}[Sir Michael Atiyah]
	Algebra is the offer made by the devil to the mathematician. The devil says: ``I will give you this powerful machine, and it will answer any question you like. All you need to do is give me your soul: give up geometry and you will have this marvellous machine.'' \textellipsis the danger to our soul is there, because when you pass over into algebraic calculation, essentially you stop thinking: you stop thinking geometrically, you stop thinking about the meaning.
\end{displayquote}

\section{Introduction}
It is with this provocative quote that we begin our discussion: indeed, algebra provides a powerful tool by which we can study, well, geometry. In particular, our goal here will be to furnish a proof of Hilbert's Nullstellensatz, which essentially states that there is a one-to-one correspondence between geometric objects (varieties) to ideals of polynomial rings. Thus if we want to study geometry, we turn to this powerful machine that is algebra\textellipsis. Some of the following material is based on \cite{artin}, but the primary references were \cite{reid} and \cite{smith}, with \cite{aluffi} as well. But first, we'll discuss a baby version of our end goal. 

\begin{proposition}\label{prop:monicirr}
	Let \(F\) be a field. The maximal ideals of \(F[x]\) are the principal ideals generated by the monic irreducible polynomials. 
\end{proposition}
\begin{proof}
	Hey, we proved this in February 17, 2022 lecture! (in particular, $F[x]$ is a PID).
\end{proof}

\begin{corollary}
	There is a bijective correspondence between the maximal ideals of \(\CC[x]\) and points in \(\CC{}\). The maximal ideal \(M_a\) that corresponds to a point \(a\in\CC\) is the kernel of the substitution homomorphism which sends \(x\) to \(a\). Indeed, \(M_a=\langle x-a\rangle\). 
\end{corollary}

\begin{proof}
	The kernel of the substitution homomorphism is all polynomials which have \(a\) as a root. So, they must be divisible by \(x-a\), and \(M_a=\langle x-a\rangle\). On the other hand, if \(M\) is a maximal ideal of \(\CC[x]\), then by \hyperref[prop:monicirr]{Proposition \ref*{prop:monicirr}} we know \(M\) is generated by the monic irreducible polynomials of \(\CC[x]\), which are simply \(x-a\). 
\end{proof}

%\noindent\rule{\textwidth}{1pt}

% The next theorem is a generalization of this fact. 

% \begin{theorem}[Hilbert's Nullstellensatz]
% 	The maximal ideals of the ring \(\CC[x_1,\dots,x_n]\) are in bijective correspondence with points of \(\CC^n\). This correspondence sends a point \(a=(a_1,\dots,a_n)\in\CC^n\) to the kernel \(M_a\) of the substitution homomorphism which sends \(x_i\) to \(a_i\). Indeed, \(M_a=\langle x_1-a_1,\dots,x_n-a_n\rangle\).  
% \end{theorem}

% \begin{proof}[Proof (Sketch)]
% 	The first part of the theorem is an extension of the last proof. The harder part is showing that all maximal ideals have the form described in \(M_a\). The rest of the proof can be found in Artin. 
% \end{proof}

\section{Affine Algebraic Sets}

\begin{definition}
    Suppose $\{f_i\}_{i\in I}$ is an indexed collection, not necessarily finite or countable, of polynomials. The common zero set \[\VV(\{f_i\}_{i\in I})\defeq \{x\in \CC^n\mid f_i(x)=0\text{ for all }i\in I\} \] is called an \textbf{affine algebraic set}.
	%We call a subset \(V\subset\CC^n\) an \textbf{algebraic variety} if it is the set of common zeros of a finite number of polynomials in \(\CC[x_1,\dots,x_n]\).  
\end{definition}

\begin{example}
	By definition, a complex line in \(\CC^2\) is made of the solutions to the equation \(ax+by+c=0\), so it is an affine algebraic set. Similarly, the point \((a,b)\) is an affine algebraic set since it is the common solution of the equations \(x-a\) and \(y-b\). The empty set is the solution of the constant function, $\varnothing=\VV(1)$, and the whole space is the solution of the constant function $\CC^2=\VV(0)$, and so are both affine algebraic sets.
\end{example}
\begin{example}
    We can identify $\Mat_{n\times n}(\CC)$ with $\CC^{n\times n}$. $\det:\Mat_{n\times n}\to \CC$ is a polynomial function, and so \[\SL(n,\CC)=\{A\in\Mat_{n\times n}(\CC)\mid \det A-1=0\}\] is an affine algebraic set. 
\end{example}
\begin{remark}
Note that in the standard Euclidean topology for $\CC^n$, every polynomial is continuous; hence for any polynomial $p$, $p^{-1}(\{0\})$ is closed since a singleton is closed in the standard topology. Then every affine algebraic set is an intersection of closed sets, and thus is closed. This discussion shows that $\GL(n,\CC)$, which is not closed, is not an affine algebraic set for all $n$. 
\end{remark}

\section{Hilbert's Basis Theorem}
\begin{definition}[Noetherian Ring]
    We say that a ring $R$ is \textbf{Noetherian} if every ideal $I\ideal R$ is finitely generated; that is, if $I\ideal R$, then there exists $f_1,\dots, f_k\in I$ such that $I=\abrackets{f_1,\dots,f_k}$.
\end{definition}
\begin{proposition}
    Let $R$ be a ring. The following are equivalent: 
    \begin{enumerate}[label=(\roman*)]
        \item $R$ is Noetherian. 
        \item $R$ satisfies the ascending chain condition (a.c.c): if $I_1\subseteq I_2\subseteq \cdots$ is an increasing chain of ideals, then the chain terminates; that is, there exists an $n$ such that $I_{n-1}\subseteq I_n=I_{n+1}=\cdots$.
        \item Every nonempty collection of ideals of $R$ has a maximal element ordered by inclusion. 
    \end{enumerate}
\end{proposition}
\begin{proof}
    ((i) $\Rightarrow$ (ii)) Let $I_1\subseteq I_2\subseteq\cdots$ be an ascending chain of ideals, and set $I\defeq \bigcup_{j\in\NN} I_j$. This is an ideal by Homework 1, and so it is finitely generated, say $I=\abrackets{f_1,\dots,f_k}$. Then each $f_i$ must be in some $I_{m_i}$ for some $m_i$, and take $m\defeq \max\{m_i\mid 1\le i\le k\}$, we have that $I=I_m$, and so the chain terminates after $m$. 
    
    ((ii) $\Rightarrow$ (iii)) For the contrapositive, suppose that $R$ has a family $\mathcal{F}$ of ideals without a maximal element. Then pick $I_1\in \mathcal{F}$; since this is not a maximal element, then there exists an $I_2\in\mathcal{F}$ such that $I_1\subset I_2$ a proper inclusion. Now inductively we may define, for $I_n\in\mathcal{F}$, an $I_{n+1}\in \mathcal{F}$ such that $I_n\subset I_{n+1}$ since $I_n$ is not maximal in $\mathcal{F}$; then we have an ascending chain $I_1\subset I_2\subset \cdots \subset I_n\subset I_{n+1}\subset \cdots$ that does not terminate. 
    
    ((iii) $\Rightarrow$ (i)) Let $I\ideal R$ be an ideal, and let $\mathcal{F}=\{J\subseteq I\mid J\text{ is a finitely generated ideal}\}$. Then certainly $\{0\}\in\mathcal{F}$ so $\mathcal{F}\ne\varnothing$. Thus $\mathcal{F}$ must have a maximal element by inclusion, call it $M$. Then $M=I$, since if not, there is an element $f\in I\setminus M$, and $M+Rf$ must be another finitely generated element which strictly contains $M$, a contradiction. 
\end{proof}

\begin{example}
    Every field is a Noetherian ring, since it only has two ideals, namely the trivial ideal and the whole field. 
\end{example}
\begin{example}
    Every PID is a Noetherian ring since every ideal is generated by a single element. In particular, $F[x]$ is a Noetherian ring if $F$ is a field.
\end{example}
Now we are ready to prove the main theorem of this section. \begin{theorem}[Hilbert's Basis Theorem]
    If $R$ is a Noetherian ring, then $R[x]$ is also Noetherian. 
\end{theorem}
\begin{proof}
    Let $J\ideal R[x]$ be an ideal. Our aim will be to show that $J$ is finitely generated. Define $J_n\subseteq R$ to be the set of leading coefficients of polynomials in $J$ of degree $n$: \[J_n\defeq \{a_n\in R\mid\text{ there exists }f \in J \text{ such that }f(x)= a_nx^n+a_{n-1}x^{n-1}+\cdots+a_1x+a_0\}.\] Then $J_n$ inherits its ideal property from $J$, and moreover for each $n$, $J_n\subseteq J_{n+1}$ since if $a\in J$, then there exists an $f\in J$ such that $a$ is the leading coefficient of $f$. Taking $xf(x)$, we see that $a$ can also be the leading coefficient of a polynomial of degree $n+1$. Now since $R$ is Noetherian, it satisfies a.c.c, and so there exists some $N$ such that $J_N=J_{N+1}=\cdots$. Now each $J_i$, $i\le N$, is finitely generated since $R$ is Noetherian, say $a_{i_1},\dots,a_{i_{m_1}}$ be the generators of $J_{i}$. Then associated to each $a_{i_k}$ is a function $f_{i_k}\in J$ with degree $i$ and leading coefficient $a_{i_k}$. Then we claim that $\{f_{i_k}\}$ generates $J$. Let $g\in J$, and let $\deg g=m$. Write the leading coefficient of $g$ as $b$, and then $b\in J_m$. Now take the generators in $J_m$ to write \[b=\sum c_{m'_k}a_{m'_k},\] where $m'=m$ if $m\le N$, and $m'=N$ if $m>N$. Now define \[g_1\defeq g-x^{m-m'}\cdot \sum c_{m'_k}f_{m'_k}.   \] This subtraction ``deletes" the leading coefficient of $g$, and so $\deg g_1\le \deg g-1$. Now by induction, we can write $g$ as a combination of $f_{i_k}$, which implies that they generate $j$. 
\end{proof}
\begin{corollary}\label{cor3.6}
    If $R$ is a Noetherian ring, then $R[x_1,\dots,x_n]$ is also Noetherian. In particular, $\CC[x_1,\dots,x_n]$ is Noetherian for all $n$. 
\end{corollary}
\begin{proof}
    Take the above and throw induction at it.
\end{proof}

\section{The Correspondences \texorpdfstring{$\VV$}{V} and \texorpdfstring{$\II$}{I}}
Now with similar notation as in Section 2, we may refine our notion of what $\VV$ is: it is a function that takes sets of polynomials and maps it to subsets of $\CC^n$. On the other hand, we define the ``inverse map" $\II$, if $V$ is a subset of $\CC^n$, not necessarily an affine algebraic set, \[\II(V)\defeq \{f\in\CC[x_1,\dots,x_n]\mid f(x)=0\text{ for all } x\in V\}.\] Clearly $\II(V)$ is an ideal. 

Next, we ask the question if $\II$ and $\VV$ are inverses; note that clearly $V\subseteq \VV(\II(V))$ since $\II$ contains all functions vanishing on $V$, and $\VV(\II(V))$ is the vanishing set of $\II(V)$. However, the reverse inclusion may not hold, and this brings us to the next proposition: \begin{proposition}
    If $V\subseteq \CC$, then $V=\VV(\II(V))$ if and only if $V$ is an affine algebraic set. 
\end{proposition} \begin{proof}
    Suppose $V=\VV(\II(V))$. Then $V$ is the image of some ideal under $\VV$, which is by definition an affine algebraic set. Conversely, suppose $V$ is an algebraic variety. We have shown one inclusion always holds; we need to show the reverse inclusion. If $x\in \VV(\II(V)),$ then $f(x)=0$ for all $f\in\II(V)$, but $V$ is the vanishing set of all $\{f_i\}_{i\in I}$, and so each $f_i\in \II(V)$ and it follows that $x\in V$, and so $\VV(\II(V))\subseteq V$, and the equality follows.
\end{proof} 

Now, from \hyperref[cor3.6]{Corollary \ref*{cor3.6}} we have that $\CC[x_1,\dots,x_n]$ is a Noetherian ring, and so if $V$ is an affine algebraic set in $\CC^n$, $\II(V)$ is an ideal and so it is \textit{finitely generated}. Therefore we have that every affine algebraic set is the zero set of finitely many polynomials. Thus every affine algebraic set can be described by finitely many polynomials, a rather surprising and powerful fact. 

Our discussion above gives us a set to which the restriction of $\VV\circ \II$ is the identity map: namely, the collection of all affine algebraic sets. Now we consider the natural follow-up question: are there necessary and sufficient conditions for which $\II\circ \VV$ is the identity map? Certainly for any ideal $I$, one has $I\subseteq \II(\VV(I))$, but this inclusion may be a strict one. Let $f\in\CC[x_1,\dots,x_n]$ and consider $f^a$ for $a\ge 2$. Then one has $f(x)=0$ if and only if $f(x)^a=0$, which implies $\VV(f)=\VV(f^a)$, which would imply that $f\in \II(\VV(f^a))$, but in general $f\notin \abrackets{f^a}$. Our hopes are now dashed, and this motivates the need for Hilbert's Nullstellensatz. 

\section{Hilbert's Nullstellensatz}
The proof of Hilbert's Nullstellensatz is a bit long, so one would naturally ponder if its inclusion is necessary especially since the soft page count limit of 2--3 has already been violated. But for the sake of completeness the authors feel compelled to discuss it, and to do that a definition is necessary. \begin{definition}
    Let $I\ideal R$. Then the \textbf{radical} of $I$ is defined as \[\rad I\defeq \setbraces{f\in R\mid \text{there exists some }n\text{ such that } f^n\in I}.\] An ideal $I$ is called a \textbf{radical ideal} if $\rad I = I$. 
\end{definition}
\begin{proposition}
    For $I\ideal R$, $\rad I$ is an ideal.
\end{proposition} \begin{proof} First let $f\in\rad I$, and say $f^n\in I$. Then if $h\in R$, $(fh)^n=h^nf^n\in I$ and so $fh\in \rad I$. If $f,g\in \rad I$, say $f^n,g^m\in I$, then by the binomial theorem \[(f+g)^r=\sum_{a=0}^r \binom{r}{a}f^ag^{r-a},\] and from here it is clear that for sufficiently large $r$ one has $(f+g)^r\in \rad I$. 
\end{proof} \begin{example}
    Every prime ideal is a radical ideal. Obviously $I\subseteq \rad I$, and conversely suppose $f\in \rad I$. Then there exists an $n$ such that $f^n\in I$, and since $f^n=ff^{n-1}$, either $f\in I$ or $f^{n-1}\in I$, and proceed inductively. 
\end{example}
\begin{example}\label{ex:I-is-radical}
	If \(V\) is an affine algebraic set, then \(\II(V)\) is a radical ideal, since if \((f(x))^n=0\) then \(f(x)=0\). So given any \(f^n\) which vanishes, we know \(f\) vanishes as well. 
\end{example}
Now we are ready to state and prove the main theorem. \begin{theorem}[Hilbert's Nullstellensatz]\label{Nullstellensatz} $~$
    \begin{enumerate}[label=(\roman*)]
        \item Every maximal ideal of the polynomial ring $\CC\brackets{x_1,\dots,x_n}$ is of the form $\abrackets{x_1-a_1,\dots, x_n-a_n}$ for some $a=(a_1,\dots,a_n)\in\CC^n$. Thus there is a one-to-one correspondence \[\{\text{maximal ideals of }\CC[x_1,\dots,x_n] \}\longleftrightarrow \{\text{points of }\CC^n\}.\]

        \item If $I\ideal \CC[x_1,\dots,x_n]$ is an ideal, then $\VV(I)=\varnothing$ if and only if $I=\abrackets{1}$. 
        \item For any $I\ideal \CC[x_1,\dots,x_n]$, \[\II(\VV(I))=\rad I.\]
    \end{enumerate}
\end{theorem}
\begin{remark}
    The word ``Nullstellensatz" is German for ``theorem of zeros." Item (ii) is sometimes referred to as the weak Nullstellensatz, and item (iii) is sometimes referred to as the strong Nullstellensatz. However,
    \begin{quote}
        \textit{``stick to the German if you don't want to be considered an ignorant peasant"} \cite{reid}. 
    \end{quote}
	We can also replace \(\CC{}\) with any algebraically closed field \(k\), but for our sakes we will just consider \(\CC\). 
\end{remark}

We will first state a lemma which will be crucial in our proof of (i), but in the interest of brevity and clarity we will not prove it. \begin{lemma}[Zariski]\label{Zariski} $~$
			\begin{enumerate}[label=(\roman*)]
				\item Let \(R\) be a ring that has \(\CC{}\) as a subring. The laws of composition on \(R\) can be used to make \(R\) into a complex vector space. 
				\item As a vector space, the field \(\mathcal F=\CC[x_1,\dots,x_n]/M\) is spanned by a countable set of elements. 
				\item Let \(V\) be a vector space spanned by a countable set of vectors. Then every independent subset of \(V\) is either finite or countably infinite. 
				\item Taking \(\CC(x)\) as a vector space over \(\CC{}\), the uncountable set of rational functions \((x-\alpha)^{-1}\) with \(\alpha\) in \(\CC\) is independent. 
			\end{enumerate}
		\end{lemma}
Now assuming the lemma, we will furnish a proof for \hyperref[Nullstellensatz]{Theorem \ref*{Nullstellensatz}}.
\begin{proof}
    We first prove (i). Let \(s_a:\CC[x_1,\dots,x_n]\to \CC\) be the substitution homomorphism of \(a\), and \(M_a\) be the kernel of \(s_a\). We know that \(s_a\) is surjective, so since \(\CC{}\) is a field \(M_a\) must be a maximal ideal. To verify that every maximal ideal is of the form in (a), we can see that if \(a=(0,0,\dots,0)\) then every monomial that appears in \(f\in s_0\) must be divisible by at least one of the variables, so \(f\) can be written as a linear combination of the variables with polynomial coefficients. For the general case we can simply make the variable substitution \(x_i=x_i'+a_i\) to move \(a\) to the origin. 
		
	The harder part is showing that all maximal ideals have the form described in \(M_a\). Let \(M\) be a maximal ideal, and \(\mathcal F=\CC[x_1,\dots,x_n]/M\). If we restrict the canonical map \(\pi:\CC[x_1,\dots,x_n]\to \mathcal F\) to the subring \(\CC[x_1]\), then we get a homomorphism \(\phi_1:\CC[x_1]\to\mathcal F\). Then the kernel of \(\phi_1\) is either the zero ideal or one of the maximal ideals \(\langle x_1-a_1\rangle\) of \(\CC[x_1]\). (Note that we could've easily swapped out \(x_1\) for any such \(x_i\).) This will mean that \(M\) contains some \(M_a\), and since \(M_a\) is maximal then \(M=M_a\). Seeking a contradiction, suppose \(\ker\phi=\langle 0\rangle\). Then we can show \(\mathcal F\) contains a field isomorphic to \(\CC(x)\). 
	
	Now we use \hyperref[Zariski]{Lemma \ref*{Zariski}}. Parts (ii) and (iii) of \hyperref[Zariski]{Lemma \ref*{Zariski}} show every independent set of \(\mathcal F\) is finite or countably infinite. But we showed \(\mathcal F\) contains a subfield isomorphic to \(\CC(x)\), so by (iv) \(\mathcal F\) contains an uncountable independent set, a contradiction. 
		
	Now we will show that (i) implies (ii). We showed that \(\VV(1)=\varnothing\) earlier. For the other direction, assume \(I\neq \langle 1\rangle\). Then there exists some maximal ideal \(m\) of \(\CC[x_1,\dots,x_n]\) such that \(I\subset m\) from the a.c.c. From (a) we know that \[m=\langle x_1-a_1,\dots,x_n-a_n\rangle,\] so clearly \(f(a)=0\) for all \(f\in I\), and thus \(a\in V(I)\). 

	Now we will finally show that (ii) implies (iii). Since we know by definition \(I\subseteq\II(\VV(I))\), by \hyperref[ex:I-is-radical]{Example \ref*{ex:I-is-radical}} we conclude \(\rad I\subseteq \rad \II(\VV(I))=\II(\VV(I))\). To show the reverse direction, we use a ``fiendishly clever'' argument known as the Rabinowitsch trick, named after the original name of George Yuri Rainich. Take some \(f\in\II(\VV(I))\). Now, introduce another variable \(y\) and consider the ideal \[I_1=\abrackets{I,fy-1}\ideal\CC[x_1,\dots,x_n,y].\] We claim that \(\VV(I_1)=\varnothing\), since any point \(q=(a_1,\dots,a_n,b)\) in \(\VV(I_1)\) must have that \((a_1,\dots,a_n)\in \VV(I)\). But this means that \((fy-1)(q)=f(a_1,\dots,a_n)b-1=-1\), which implies that \(q\not\in\VV(I_1)\), a contradiction. 

	Then by (b), we know that \(I_1=\abrackets{1}\). So there exist \(f_i\in I\), \(g_0,g_1\in \CC[x_1,\dots,x_n,y]\) such that \[1=\sum g_if_i+g_0(fy-1).\] Viewing this equality as an identity in the field of fractions \(\CC(x_1,\dots,x_n,y)\), we can let \(y=f^{-1}\) and see that \[1=\sum g_i(x_1,\dots,x_n,f^{-1})f_i(x_1,\dots,x_n,f^{-1}).\] But \(g_i\in I\ideal\CC[x_1,\dots,x_n]\), so the extra \(y\) variable does not matter to them, and \[f_i=\frac{\hat{f}_i}{f^m},\] for some \(\hat{f}_i\in\CC[x_1,\dots,x_n]\), for some sufficiently large integer \(m\). Then we can multiply out by \(f^m\) to see that \[f^m=\sum g_i\hat{f}_i\in I\] meaning \(f\in\rad I\), as desired. 
\end{proof}
\begin{quote}
    \textit{``This requires a cunning trick" \cite{reid}.}
    
    \textit{``The argument is not difficult \textellipsis but it is fiendishly clever" \cite{aluffi}.}
\end{quote}

\begin{corollary}
    The correspondences $\VV$ and $\II$ induce bijections \[\{\text{radical ideals}\}\longleftrightarrow \{\text{algebraic subsets}\}.\]
\end{corollary}

\bibliographystyle{apalike}

\bibliography{alggeoref}

% It turns out that the Nullstellensatz gives us a connection between algebraic varieties and quotient rings of the polynomial ring. This is where the name of the theorem comes from (in German, it means ``theorem of zeros.'').

% \begin{theorem}
% 	Let \(I\) be the ideal of \(\CC[x_1,\dots,x_n]\) generated by some polynomials \(f_1,\dots,f_r\), and let \(R\) be the quotient ring \(\CC[x_1,\dots,x_n]/I\). and let \(V\) be the variety given by these polynomials in \(\CC^n\). Then the maximal ideals of \(R\) are in bijective correspondence with the points of \(V\). 
% \end{theorem}

% \begin{proof}
% 	Note by the Correspondence Theorem we have that the maximal ideals of \(R\) are in one-to-one correspondence with the maximal ideals of \(\CC[x_1,\dots,x_n]\) that contain \(I\). Then a maximal ideal will contain \(I\) if and only if it contains the generators \(f_1,\dots,f_r\) of \(I\). But notice that every maximal ideal of \(\CC[x_1,\dots,x_n]\) is the kernel \(M_a\) of the substitution map \(x_i\mapsto a_i\) for some point \(a=(a_1,\dots,a_n)\). This means that the polynomials \(f_1,\dots,f_r\in M_a\) if and only if \(f_1(a)=\dots=f_n(a)=0\), i.e., if \(a\) is in the variety \(V\). 
% \end{proof}

% % there's a whole page at this point in Artin with a proof 
% % of a theorem involving AC and some corollaries of said theorem, 
% % but I'm not including it here because it doesn't
% % seem to have a connection to the rest of the section, and we
% % don't have a ton of room. we can put it back later if needed

% It's hard to visualize varieties in \(\CC^n\), so for the rest of this exposition, we will turn our focus to the special case of \(\CC^2\), and more specifically the polynomial ring \(\CC[t,x]\). 

% \begin{lemma}
% 	Let \(f(t,x)\) be a polynomial, and \(\alpha\in\CC{}\). Then the following are equivalent: 
% 	\begin{enumerate}[label=\emph{(\alph*)}]
% 		\item \(f(t,x)\) vanishes everywhere on the locus \(\{t=\alpha\}\) in \(\CC^2\). 
% 		\item \(f(\alpha,x)\) is the zero polynomial in \(x\). 
% 		\item \(t-\alpha\) divides \(f\) in \(\CC[t,x]\). 
% 	\end{enumerate}
% \end{lemma}
% \begin{proof}
% 	blah djshfjkhds
% \end{proof}

% Let \(\mathcal F\) be the field of rational functions \(\CC(t)\). Note that \(\CC[t,x]\) is a subring of \(\mathcal F[x]\), as we can write 
% \[f(t,x)=a_n(t)x^n+\cdots+a_1(t)x+a_0(t).\] Whenever we think about \(\CC[t,x]\), it is often helpful to investigate \(\mathcal F\) instead since it is a PID and we have access to the division algorithm. 

% \begin{proposition}
% 	Let \(h,f\) be non-zero elements of \(\CC[t,x]\) such that \(h\) is not divisible by any polynomial of the form \(t-\alpha\). Then \(h\mid f\) in \(\mathcal F[x]\) implies that \(h\mid f\) in \(\CC[t,x]\). 
% \end{proposition}
% \begin{proof}
% 	blahjwskdjskjdks
% \end{proof}

% \begin{theorem}
% 	Two non-zero polynomials \(f,g\in\CC[t,x]\) have finitely many zeros in \(\CC^2\), unless they have a common non-constant factor in \(\CC[t,x]\). 
% \end{theorem}
% \begin{proof}
% 	jldsfsjkfdsbksjfdb
% \end{proof}

% This theorem suggests that the varieties worth investigating in \(\CC^2\) are the locus of zeros of a single polynomial \(f(t,x)\). 

% % SANGJUN READ THIS 
% % I also doubt that the proofs above are particularly
% % relevant to the Riemann surface bs (the ones where)
% % I spammed in the proofs, i'm considering just
% % not including them
% % Read. -SK

% \begin{definition}
% 	The locus of zeros in \(\CC^2\) of a single polynomial \(f(t,x)\) is called the \textbf{Riemann surface} of \(f\). 
% \end{definition}

\end{document}