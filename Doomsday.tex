\documentclass[12pt,letterpaper]{article}
\usepackage{fullpage}
\usepackage[top=2cm, bottom=4.5cm, left=2.5cm, right=2.5cm]{geometry}
\usepackage{amsmath,amsthm,amscd}
\usepackage{amsfonts}
\usepackage{amssymb}
\usepackage{lastpage}
\usepackage{enumerate}
\usepackage{fancyhdr}
\usepackage{xcolor}
\usepackage{graphicx}
\usepackage{listings}
\usepackage{hyperref}
\usepackage{enumitem}
\usepackage{XCharter}
\usepackage[T1]{fontenc}
\usepackage{textcomp}
\usepackage[euler-hat-accent]{eulervm}
\usepackage{eucal}
\usepackage{beton}
%\DeclareFontSeriesDefault[rm]{bf}{sbc}
\usepackage{xparse}
%\usepackage{comicsans}

\usepackage{polynom}

\usepackage{bm}
\usepackage[b]{esvect}

\usepackage{tikz}
\usepackage{physics}
\usepackage{tikz-cd}
\usepackage{faktor}

\hypersetup{%
  colorlinks=true,
  linkcolor=blue,
  linkbordercolor={0 0 1}
}

\setlength{\parindent}{0.0in}
\setlength{\parskip}{0.05in}

% Edit these as appropriate
\newcommand\course{Math 549}
\newcommand\hwnumber{11}                  % <-- homework number
\newcommand\studentname{Stephen Hu}           % <-- NetID of person #1
\newcommand\NetIDb{RUID: 192000354}           % <-- NetID of person #2 (Comment this line out for problem sets)

\pagestyle{fancyplain}
\headheight 30pt
\lhead{\studentname}
\lhead{\studentname\\\NetIDb}                 % <-- Comment this line out for problem sets (make sure you are person #1)
\chead{\textbf{\Large Problem Set \hwnumber{}}}
\rhead{\course \\ \today}
\lfoot{}
\cfoot{}
\rfoot{\small\thepage}
\headsep 1.5em
\linespread{1.25} % line spacing

\newtheorem{theorem}{Theorem}
\newtheorem{definition}{Definition}
\newtheorem*{remark}{Remark}
\newtheorem{lemma}[theorem]{Lemma}

\newcommand\ZZ{\mathbb{Z}}
\newcommand\NN{\mathbb{N}}
\newcommand\RR{\mathbb{R}}
\newcommand\QQ{\mathbb{Q}}
\newcommand\CC{\mathbb{C}}
\renewcommand\aa{\alpha}
\newcommand\bb{\beta}
\newcommand\cc{\gamma}
\newcommand\card{\operatorname{card}}
%\newcommand\rank{\operatorname{rank}}
\newcommand\im{\operatorname{im}}
\newcommand\sgn{\operatorname{sgn}}
%\newcommand\tr{\operatorname{tr}}
\newcommand\Span{\operatorname{span}}
\newcommand\Aut{\operatorname{Aut}}
\newcommand\Hom{\operatorname{Hom}}

\renewcommand{\vec}[1]{\vv{\boldsymbol{#1}}}

\ExplSyntaxOn
\NewDocumentCommand{\cycle}{ O{\;} m }
 {
  (
  \alec_cycle:nn { #1 } { #2 }
  )
 }

\seq_new:N \l_alec_cycle_seq
\cs_new_protected:Npn \alec_cycle:nn #1 #2
 {
  \seq_set_split:Nnn \l_alec_cycle_seq { , } { #2 }
  \seq_use:Nn \l_alec_cycle_seq { #1 }
 }
\ExplSyntaxOff

\begin{document}

\section*{Problem 1}

Note that \(\gamma:[0,2\pi]\to SO(3)\) is defined by \[\gamma(\theta)=\begin{pmatrix}
	R_\theta & 0 \\
	0 & 1
\end{pmatrix},\] where \(R_\theta\) is the rotation matrix by \(\theta\). Then \(\gamma^2=\gamma(2\theta)\). Now if let \(\Gamma:[0,1]\times[0,1]\to SO(3)\) be the function defined by 
\[\Gamma(s, t)=\begin{cases}
	\exp(At) & t\in\left[0,\frac{1}{2}\right] \\
	B(s)\exp(At)B(s)^{-1} & t\in\left[\frac{1}{2},1\right] 
\end{cases},\]
where \[
A=\begin{pmatrix}
	0 & -4\pi & 0 \\
	4\pi & 0 & 0 \\
	0 & 0 & 0
\end{pmatrix},	
B(s)=\begin{pmatrix}
	\dmat{1,R_{\pi+\pi s}}
\end{pmatrix}.\]
Then it can be easily verified that this is a homotopy, since \(\Gamma(s,0)=\Gamma(s,1)=I_3\), \(\Gamma(1,t)=\gamma^2\), and \(\Gamma(0,t)\) is the trivial loop. In particular, \(\exp(At)=\gamma^2(t)\), meaning that \[B(0)\exp(At)B(0)^{-1}=\exp(-At).\] 

\section*{Problem 2}

Let \(\beta\) be a positive root different from \(\alpha\). Then we can write by definition \[\beta=\sum_{\gamma\in S} m(\gamma)\gamma,\] where \(S\) is the simple root set and \(m(\gamma)\) are all non-negative. Indeed, since \(\beta\neq\alpha\) we have that \(m(\gamma)>0\) for some \(\gamma\). But recall that \(s_\alpha(\beta)=\beta-n_{\alpha,\beta}\alpha\), so if we represent \(s_\alpha(\beta)\) in terms of the basis \(S\), the only coefficients which change is the \(\alpha\) coefficient. In particular this means the \(\gamma\) from earlier is still positive, which means that \(s_\alpha(\beta)\) must be another positive root. 

\end{document}